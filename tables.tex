\documentclass{article}
\usepackage[utf8]{inputenc}

\title{Faktorenanalyse Q30a-Q30f}

\date{}

\begin{document}

\maketitle
\section{Factor loadings (pattern matrix) and unique variances}
\centering
{
\def\sym#1{\ifmmode^{#1}\else\(^{#1}\)\fi}
\begin{tabular}{l*{1}{ccc}}
  \hline\hline
 ID & Country Code & XX& Count of Observations \\ 
  \hline
1 & V149 & Claiming state benefits which you are not entitled to \\ 
  2 & V150 & Cheating on tax if you have the chance \\ 
  3 & AUS & Taking the drugs marijuana or hashish \\ 
  4 & BGD & Someone accepting a bribe in the course of their duties \\ 
  5 & BOL & Homosexuality \\ 
  6 & BRA & Abortion \\ 
  7 & CHL & Divorce \\ 
  8 & CHN & Euthanasia \\ 
  9 & COL & Suicide \\ 
  10 & CYP & Having casual sex \\ 
  11 & DEU & Avoiding a fare on public transport\\ 
  12 & ECU & Prostitution \\ 
  13 & ETH & Political violence \\ 
  14 & GRC & Death penalty \\ 
  \hline\hline
\end{tabular}
}
\bigskip
\bigskip
\bigskip

\section{Rotated factors loadings (pattern matrix) and unique variances}
\centering
{
\def\sym#1{\ifmmode^{#1}\else\(^{#1}\)\fi}
\begin{tabular}{l*{1}{ccc}}
\hline\hline
            &    Factor 1&    Factor 2&  Uniqueness\\
\hline
q30a        &    .6159174&    .1848204&    .5864872\\
q30b        &    .5760492&    .2133746&    .6226386\\
q30c        &    .4019481&    .4569902&    .6295977\\
q30d        &    .4518697&    .2592333&    .7286119\\
q30e        &    .3083852&    .5051477&    .6497244\\
q30f        &    .1196099&     .439956&    .7921322\\
\hline\hline
\end{tabular}
}

\end{document}

 \begin{table}[htb]
  \centering
    \begin{threeparttable}
      \caption{Results Cluster Analysis}
       \centering  %% use this instead of \begin{center}
        \begin{tabular}{l c c c c c}
          \toprule
          {Algorithm} & {Optimal number of clusters} & {Cluster 1} & {Cluster 2} & {Cluster 3}& {Cluster 4} \\\toprule[1pt]
          K-Means & 4 &13 &21 &23& 15\\
          Hierarchical & 4 &13 &21 &23& 15\\
          \bottomrule
        \end{tabular}
        \begin{tablenotes}
          %\footnotesize   %% If you want them smaller like foot notes
          \item[a] Note: To determine the right amount of clusters, I relied on XX and XX methods for K-Means algorithm and XX and XX for Hierarchical algoritms
        \end{tablenotes}
    \end{threeparttable}
  \end{table}